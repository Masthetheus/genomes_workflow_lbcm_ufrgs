% Module: BUSCO Analysis for Genome Assessment
% Author: LBCM Course Team
% Date: 2024

\section{BUSCO Analysis for Genome Assessment}

\subsection{Introduction to BUSCO}

BUSCO (Benchmarking Universal Single-Copy Orthologs) is a widely used tool for assessing genome completeness and annotation quality. It searches for highly conserved genes that are expected to be present in most species within a phylogenetic group.

\subsubsection{Key Concepts}
\begin{itemize}
    \item \textbf{Universal Single-Copy Orthologs}: Genes that are conserved across species and typically present in single copy
    \item \textbf{Completeness Assessment}: Measure of how complete a genome assembly or annotation is
    \item \textbf{Lineage-Specific Datasets}: Gene sets tailored to specific phylogenetic groups
\end{itemize}

\subsection{BUSCO Installation and Setup}

\subsubsection{Prerequisites}
Before installing BUSCO, ensure you have:
\begin{itemize}
    \item Python 3.3 or higher
    \item Conda package manager
    \item Sufficient disk space for databases
\end{itemize}

\subsubsection{Installation via Conda}
The recommended installation method uses conda:

\begin{verbatim}
# Update conda
conda update -n base conda

# Add bioconda channel
conda config --add channels bioconda
conda config --add channels conda-forge
conda config --set channel_priority strict

# Install BUSCO
conda install -c bioconda busco=5.4.7
\end{verbatim}

\subsubsection{Verification}
Test the installation:
\begin{verbatim}
busco --help
busco --version
\end{verbatim}

\subsection{BUSCO Datasets}

\subsubsection{Available Lineages}
List available datasets:
\begin{verbatim}
busco --list-datasets
\end{verbatim}

Common lineages include:
\begin{itemize}
    \item \textbf{bacteria\_odb10}: Bacterial genomes
    \item \textbf{archaea\_odb10}: Archaeal genomes
    \item \textbf{eukaryota\_odb10}: Eukaryotic genomes
    \item \textbf{fungi\_odb10}: Fungal genomes
    \item \textbf{metazoa\_odb10}: Animal genomes
    \item \textbf{embryophyta\_odb10}: Plant genomes
\end{itemize}

\subsubsection{Dataset Download}
Datasets are automatically downloaded when first used, but can be pre-downloaded:
\begin{verbatim}
busco --download-dataset bacteria_odb10
\end{verbatim}

\subsection{Running BUSCO Analysis}

\subsubsection{Basic Usage}
The general BUSCO command structure:
\begin{verbatim}
busco -i INPUT -l LINEAGE -o OUTPUT -m MODE [OPTIONS]
\end{verbatim}

\subsubsection{Parameters}
\begin{itemize}
    \item \textbf{-i}: Input sequence file (FASTA format)
    \item \textbf{-l}: Lineage dataset to use
    \item \textbf{-o}: Output directory name
    \item \textbf{-m}: Mode (genome, transcriptome, or proteins)
    \item \textbf{-c}: Number of CPU cores to use
    \item \textbf{--force}: Overwrite existing output
\end{itemize}

\subsubsection{Example Commands}

\textbf{Genome Assembly Assessment:}
\begin{verbatim}
busco -i genome.fasta -l bacteria_odb10 -o busco_genome -m genome -c 8
\end{verbatim}

\textbf{Protein Annotation Assessment:}
\begin{verbatim}
busco -i proteins.faa -l bacteria_odb10 -o busco_proteins -m proteins -c 8
\end{verbatim}

\textbf{Transcriptome Assessment:}
\begin{verbatim}
busco -i transcripts.fasta -l bacteria_odb10 -o busco_transcripts -m transcriptome -c 8
\end{verbatim}

\subsection{Understanding BUSCO Output}

\subsubsection{Output Files}
BUSCO creates several output files:
\begin{itemize}
    \item \textbf{short\_summary.txt}: Concise results summary
    \item \textbf{full\_table.tsv}: Detailed results for each BUSCO
    \item \textbf{missing\_busco\_list.tsv}: List of missing BUSCOs
    \item \textbf{run\_info/}: Detailed run information
\end{itemize}

\subsubsection{Result Categories}
BUSCO classifies genes into categories:
\begin{itemize}
    \item \textbf{Complete (C)}: BUSCO found completely
    \item \textbf{Complete Single-copy (S)}: Found once (expected)
    \item \textbf{Complete Duplicated (D)}: Found multiple times
    \item \textbf{Fragmented (F)}: Partial matches found
    \item \textbf{Missing (M)}: No match found
\end{itemize}

\subsubsection{Quality Metrics}
\begin{itemize}
    \item \textbf{Completeness}: (C + F) / Total BUSCOs
    \item \textbf{Duplication}: D / (C + F)
    \item \textbf{Fragmentation}: F / Total BUSCOs
\end{itemize}

\subsection{Interpreting BUSCO Results}

\subsubsection{Good Quality Indicators}
\begin{itemize}
    \item High completeness (>95\% for high-quality genomes)
    \item Low duplication (<5\% for most genomes)
    \item Low fragmentation (<5\%)
    \item Low missing percentage (<5\%)
\end{itemize}

\subsubsection{Example Result Interpretation}
\begin{verbatim}
Results:
    C:98.5%[S:97.8%,D:0.7%],F:0.8%,M:0.7%,n:124
    
Interpretation:
- 98.5% complete BUSCOs (excellent)
- 97.8% single-copy (good)
- 0.7% duplicated (acceptable)
- 0.8% fragmented (good)
- 0.7% missing (excellent)
\end{verbatim}

\subsection{Advanced BUSCO Usage}

\subsubsection{Custom Parameters}
\begin{verbatim}
busco -i genome.fasta \
      -l bacteria_odb10 \
      -o busco_analysis \
      -m genome \
      -c 16 \
      --long \
      --augustus \
      --augustus_species generic
\end{verbatim}

\subsubsection{Plotting Results}
Generate plots from BUSCO output:
\begin{verbatim}
generate_plot.py -wd /path/to/busco/summaries
\end{verbatim}

\subsection{Practical Exercise: BUSCO Analysis}

\subsubsection{Exercise Setup}
We'll analyze a bacterial genome using BUSCO:

\begin{enumerate}
    \item Download a sample bacterial genome
    \item Run BUSCO analysis
    \item Interpret the results
    \item Compare with reference genomes
\end{enumerate}

\subsubsection{Step-by-Step Analysis}

\textbf{Step 1: Prepare the genome file}
\begin{verbatim}
# Download sample genome (E. coli)
wget https://ftp.ncbi.nlm.nih.gov/genomes/refseq/bacteria/Escherichia_coli/latest_assembly_versions/GCF_000005825.2_ASM582v2/GCF_000005825.2_ASM582v2_genomic.fna.gz
gunzip GCF_000005825.2_ASM582v2_genomic.fna.gz
\end{verbatim}

\textbf{Step 2: Run BUSCO}
\begin{verbatim}
busco -i GCF_000005825.2_ASM582v2_genomic.fna \
      -l bacteria_odb10 \
      -o ecoli_busco \
      -m genome \
      -c 4 \
      --force
\end{verbatim}

\textbf{Step 3: Examine results}
\begin{verbatim}
# View summary
cat ecoli_busco/short_summary.txt

# View detailed results
head -20 ecoli_busco/full_table.tsv

# Check missing BUSCOs
cat ecoli_busco/missing_busco_list.tsv
\end{verbatim}

\subsubsection{Expected Results}
For a high-quality E. coli genome, expect:
\begin{itemize}
    \item Completeness: >99\%
    \item Duplication: <1\%
    \item Fragmentation: <1\%
    \item Missing: <1\%
\end{itemize}

\subsection{Troubleshooting Common Issues}

\subsubsection{Installation Problems}
\begin{itemize}
    \item \textbf{Conda conflicts}: Use dedicated environment
    \item \textbf{Missing dependencies}: Install from bioconda
    \item \textbf{Version conflicts}: Specify exact versions
\end{itemize}

\subsubsection{Runtime Issues}
\begin{itemize}
    \item \textbf{Memory errors}: Reduce thread count
    \item \textbf{Disk space}: Ensure sufficient storage
    \item \textbf{Network issues}: Pre-download datasets
\end{itemize}

\subsubsection{Result Interpretation}
\begin{itemize}
    \item \textbf{Low completeness}: Check input quality
    \item \textbf{High duplication}: Possible contamination
    \item \textbf{Wrong lineage}: Verify organism classification
\end{itemize}

\subsection{BUSCO in Genomics Workflows}

\subsubsection{Quality Control}
BUSCO is commonly used for:
\begin{itemize}
    \item Assembly quality assessment
    \item Annotation completeness evaluation
    \item Contamination detection
    \item Comparative genomics studies
\end{itemize}

\subsubsection{Publication Standards}
Many journals require BUSCO scores for:
\begin{itemize}
    \item Genome assemblies
    \item Transcriptome assemblies
    \item Protein annotations
    \item Comparative studies
\end{itemize}

\subsection{Best Practices}

\begin{enumerate}
    \item \textbf{Choose appropriate lineage}: Use most specific dataset
    \item \textbf{Consider contamination}: High duplication may indicate issues
    \item \textbf{Compare with related species}: Context matters
    \item \textbf{Document parameters}: Include in methods
    \item \textbf{Archive results}: Keep detailed outputs
\end{enumerate}

\subsection{Summary}

BUSCO is an essential tool for genome quality assessment that:
\begin{itemize}
    \item Provides standardized completeness metrics
    \item Enables comparison across studies
    \item Helps identify assembly and annotation issues
    \item Is widely accepted in the genomics community
\end{itemize}

\subsection{Next Steps}

In the next module, we will explore genome annotation pipelines that integrate BUSCO assessment with other quality control tools and annotation methods.