% Module: Introduction to Genome Annotation
% Author: LBCM Course Team
% Date: 2024

\section{Introduction to Genome Annotation}

\subsection{What is Genome Annotation?}

Genome annotation is the process of identifying and describing the functional elements within a genome sequence. It involves:

\begin{itemize}
    \item \textbf{Structural annotation}: Identifying genes, exons, introns, regulatory elements
    \item \textbf{Functional annotation}: Assigning biological functions to identified features
    \item \textbf{Comparative annotation}: Using comparative genomics to improve accuracy
\end{itemize}

\subsection{Types of Genomic Features}

\subsubsection{Protein-Coding Genes}
The most fundamental units of genetic information that encode proteins. Key characteristics:
\begin{itemize}
    \item Start and stop codons
    \item Open reading frames (ORFs)
    \item Splice sites in eukaryotes
    \item Ribosome binding sites in prokaryotes
\end{itemize}

\subsubsection{Non-Coding RNA Genes}
Important regulatory and structural RNA molecules:
\begin{itemize}
    \item Transfer RNA (tRNA)
    \item Ribosomal RNA (rRNA)
    \item MicroRNA (miRNA)
    \item Long non-coding RNA (lncRNA)
\end{itemize}

\subsubsection{Regulatory Elements}
Sequences that control gene expression:
\begin{itemize}
    \item Promoters
    \item Enhancers and silencers
    \item Terminators
    \item Binding sites for transcription factors
\end{itemize}

\subsection{Annotation Workflows}

\subsubsection{Evidence-Based Annotation}
Using experimental and computational evidence:

\begin{enumerate}
    \item \textbf{Ab initio gene prediction}: Using computational models
    \item \textbf{Homology-based methods}: Comparing to known sequences
    \item \textbf{RNA-seq data}: Using transcriptome evidence
    \item \textbf{Protein evidence}: Mapping known proteins
\end{enumerate}

\subsubsection{Quality Assessment}
Ensuring annotation accuracy:
\begin{itemize}
    \item Completeness metrics (e.g., BUSCO scores)
    \item Consistency checks
    \item Manual curation
    \item Community validation
\end{itemize}

\subsection{Common Annotation Tools}

\subsubsection{Prokaryotic Annotation}
\begin{itemize}
    \item \textbf{Prokka}: Rapid prokaryotic genome annotation
    \item \textbf{RAST}: Rapid Annotation using Subsystem Technology
    \item \textbf{NCBI PGAP}: Prokaryotic Genome Annotation Pipeline
\end{itemize}

\subsubsection{Eukaryotic Annotation}
\begin{itemize}
    \item \textbf{MAKER}: Evidence-based annotation pipeline
    \item \textbf{BRAKER}: Gene prediction with RNA-seq
    \item \textbf{Augustus}: Ab initio gene prediction
\end{itemize}

\subsection{File Formats in Genome Annotation}

\subsubsection{FASTA Format}
For sequence data:
\begin{verbatim}
>scaffold_1 length=1000000
ATGCGATCGTAGCTAGCTAGCTAGCTAG...
\end{verbatim}

\subsubsection{GFF/GTF Format}
For annotation features:
\begin{verbatim}
scaffold_1  Prokka  gene  1000  2000  .  +  .  ID=gene_001
scaffold_1  Prokka  CDS   1000  2000  .  +  0  ID=cds_001;Parent=gene_001
\end{verbatim}

\subsubsection{GenBank Format}
Comprehensive sequence and annotation:
\begin{verbatim}
LOCUS       scaffold_1    1000000 bp    DNA     linear   BCT 01-JAN-2024
DEFINITION  Bacterial genome scaffold 1
FEATURES             Location/Qualifiers
     gene            1000..2000
                     /gene="exampleGene"
     CDS             1000..2000
                     /product="hypothetical protein"
\end{verbatim}

\subsection{Practical Exercise: Basic Annotation Concepts}

\subsubsection{Exercise 1: Identifying ORFs}
Given a DNA sequence, identify potential open reading frames:

\begin{verbatim}
Sequence: ATGAAACGCTAAGCTAGCTAGCTAG...
Task: Find all ORFs longer than 300 bp
\end{verbatim}

\subsubsection{Exercise 2: Feature Annotation}
Practice annotating features in a sample sequence:

\begin{enumerate}
    \item Download a small bacterial genome
    \item Identify potential genes using ORF finding
    \item Search for homologous sequences
    \item Assign functional annotations
\end{enumerate}

\subsection{Key Concepts Summary}

\begin{itemize}
    \item Genome annotation combines computational prediction with experimental evidence
    \item Different approaches are needed for prokaryotic vs. eukaryotic genomes
    \item Quality assessment is crucial for reliable annotations
    \item Standard file formats facilitate data exchange and analysis
    \item Manual curation improves annotation quality
\end{itemize}

\subsection{Further Reading}

\begin{itemize}
    \item Yandell, M. \& Ence, D. A beginner's guide to eukaryotic genome annotation. Nat Rev Genet 13, 329–342 (2012)
    \item Salzberg, S. L. Next-generation genome annotation: we still struggle to get it right. Genome Biol 20, 92 (2019)
    \item Mudge, J. M. \& Harrow, J. Creating reference gene annotation for the mouse C57BL6/J genome assembly. Mamm Genome 26, 366–378 (2015)
\end{itemize}

\subsection{Next Steps}

In the next module, we will explore BUSCO analysis, a crucial tool for assessing genome completeness and annotation quality. This will provide practical experience with one of the most widely used quality assessment tools in genomics.