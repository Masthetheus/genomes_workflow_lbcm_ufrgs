% Genomes Workflow Module Template - LBCM UFRGS
% Based on NotesTeX and Gilles Castel's aesthetic approach
% Module: 00 - Introduction
% Date: DATE_PLACEHOLDER

\documentclass[a4paper,11pt]{article}

% Use ONLY the beautiful notes preamble for notes
% Beautiful Notes Preamble - ONLY FOR DAILY RESEARCH NOTES
% NotesTeX-inspired aesthetic design with sidenotes and colored boxes
% DO NOT use for academic papers - use paper_preamble.tex instead

% Basic packages
\usepackage[utf8]{inputenc}
\usepackage[T1]{fontenc}
\usepackage{lmodern}
\usepackage{microtype}
\usepackage[english]{babel}

% WIDE MARGINS FOR SIDENOTES (Notes-specific feature)
\usepackage[
    left=2cm,
    right=6cm,
    top=2cm,
    bottom=2cm,
    marginparwidth=4cm,
    marginparsep=0.5cm
]{geometry}

% Mathematics packages
\usepackage{amsmath}
\usepackage{amssymb}
\usepackage{amsthm}
\usepackage{mathtools}
\usepackage{physics}
\usepackage{siunitx}

% Graphics and figures
\usepackage{graphicx}
\usepackage{float}
\usepackage{caption}
\usepackage{subcaption}
\usepackage{tikz}
\usepackage{pgfplots}
\pgfplotsset{compat=1.18}

% BIBLIOGRAPHY SUPPORT FOR NOTES
\usepackage[style=authoryear,backend=biber,natbib=true]{biblatex}

% Add bibliography resources (notes can reference paper's bibliography)
\addbibresource{../references.bib}
\addbibresource{../../../global/bibliography/master.bib}

% NOTES-SPECIFIC COLORS
\usepackage{xcolor}
\definecolor{noteblue}{RGB}{0,102,204}
\definecolor{sidenotegray}{RGB}{128,128,128}
\definecolor{theoremblue}{RGB}{230,240,255}
\definecolor{definitiongreen}{RGB}{240,255,240}
\definecolor{remarkyellow}{RGB}{255,255,230}

% BEAUTIFUL THEOREM BOXES (Notes-specific feature)
\usepackage{tcolorbox}
\tcbuselibrary{theorems}

\newtcolorbox{definitionbox}{
    colback=definitiongreen,
    colframe=green!50!black,
    title=Definition,
    fonttitle=\bfseries
}

\newtcolorbox{theorembox}{
    colback=theoremblue,
    colframe=blue!50!black,
    title=Theorem,
    fonttitle=\bfseries
}

\newtcolorbox{lemmabox}{
    colback=theoremblue,
    colframe=blue!50!black,
    title=Lemma,
    fonttitle=\bfseries
}

\newtcolorbox{proofbox}{
    colback=white,
    colframe=black!20,
    title=Proof,
    fonttitle=\bfseries
}

\newtcolorbox{remarkbox}{
    colback=remarkyellow,
    colframe=orange!50!black,
    title=Remark,
    fonttitle=\bfseries
}

\newtcolorbox{examplebox}{
    colback=white,
    colframe=gray!50,
    title=Example,
    fonttitle=\bfseries
}

% SIDENOTES (Notes-specific feature)
\usepackage{marginnote}
\renewcommand*{\marginfont}{\footnotesize\color{sidenotegray}}

% Custom sidenote command
\newcommand{\sn}[1]{\marginnote{\raggedright #1}}

% Custom margin note command
\newcommand{\mn}[1]{\marginpar{\footnotesize\color{sidenotegray}\raggedright #1}}

% BEAUTIFUL HEADERS (Notes-specific styling)
\usepackage{fancyhdr}
\setlength{\headheight}{14pt}
\pagestyle{fancy}
\fancyhf{}
\fancyhead[L]{\textcolor{noteblue}{\textbf{01 - Introduction}}}
\fancyhead[R]{\textcolor{noteblue}{\today}}
\fancyfoot[C]{\textcolor{noteblue}{\thepage}}
\renewcommand{\headrulewidth}{0.5pt}
\renewcommand{\footrulewidth}{0.5pt}

% COLORFUL SECTION STYLING (Notes-specific)
\usepackage{titlesec}
\titleformat{\section}
  {\Large\bfseries\color{noteblue}}
  {\thesection}{1em}{}
\titleformat{\subsection}
  {\large\bfseries\color{noteblue}}
  {\thesubsection}{1em}{}

% Code listings with beautiful styling
\usepackage{listings}

\lstdefinestyle{notesstyle}{
    backgroundcolor=\color{gray!10},
    commentstyle=\color{green!50!black},
    keywordstyle=\color{blue},
    numberstyle=\tiny\color{gray},
    stringstyle=\color{red!50!black},
    basicstyle=\ttfamily\footnotesize,
    breakatwhitespace=false,
    breaklines=true,
    captionpos=b,
    keepspaces=true,
    numbers=left,
    numbersep=5pt,
    showspaces=false,
    showstringspaces=false,
    showtabs=false,
    tabsize=2
}
\lstset{style=notesstyle}

% Hyperlinks
\usepackage{hyperref}
\hypersetup{
    colorlinks=true,
    linkcolor=noteblue,
    citecolor=red,
    urlcolor=noteblue,
    pdftitle={Genome Workflow - Base Guide},
    pdfauthor={LBCM UFRGS},
    pdfsubject={General guidelines for working with genomes in the LBCM},
    pdfkeywords={genomics, workflow, LBCM, UFRGS}
}

% RESEARCH-SPECIFIC COMMANDS (Notes only)
\newcommand{\lecture}[2]{\section{#1 - #2}}
\newcommand{\reading}[2]{\subsection{Reading: #1}\sn{#2}}
\newcommand{\idea}[1]{\textbf{\textcolor{purple}{[IDEA]: #1}}}
\newcommand{\todo}[1]{\textbf{\textcolor{red}{[TODO]: #1}}}
\newcommand{\important}[1]{\textbf{\textcolor{red}{#1}}}
\newcommand{\highlight}[1]{\textbf{\textcolor{noteblue}{#1}}}

% Math shortcuts
\newcommand{\R}{\mathbb{R}}
\newcommand{\N}{\mathbb{N}}
\newcommand{\Z}{\mathbb{Z}}
\newcommand{\Q}{\mathbb{Q}}
\newcommand{\C}{\mathbb{C}}
\newcommand{\eps}{\varepsilon}
\newcommand{\del}{\partial}

% NOTES-SPECIFIC ENVIRONMENTS (match template exactly)
\newenvironment{progress}{
    \section{Today's Progress}
    \begin{itemize}
}{\end{itemize}}

\newenvironment{ideas}{
    \section{Ideas \& Insights}
    \begin{itemize}
}{\end{itemize}}

\newenvironment{questions}{
    \section{Research Questions}
    \begin{itemize}
}{\end{itemize}}

\newenvironment{references}{
    \section{References}
    \begin{itemize}
}{\end{itemize}}

\newenvironment{notes}{
    \section{Research Notes}
}{}

% THEOREM ENVIRONMENTS using amsthm (match template exactly)
\theoremstyle{definition}
\newtheorem{definition}{Definition}[section]
\newtheorem{theorem}{Theorem}[section]
\newtheorem{lemma}{Lemma}[section]
\newtheorem{example}{Example}[section]
\newtheorem{remark}{Remark}[section]

% Proof environment
\renewenvironment{proof}[1][\proofname]{%
    \par\pushQED{\qed}%
    \normalfont\topsep6\p@\@plus6\p@\relax
    \trivlist\item[\hskip\labelsep\bfseries #1\@addpunct{.}]\ignorespaces
}{%
    \popQED\endtrivlist\@endpefalse
}

% Date formatting
\usepackage{datetime}
\newdateformat{notedate}{\THEDAY\ \monthname[\THEMONTH] \THEYEAR}


\title{
    \vfill
    \textcolor{noteblue}{\Huge Genomes Workflow - LBCM}\\
    \vspace{5.0cm}
    \textcolor{sidenotegray}{\huge 00 - Introduction}\\
    \vspace{1.5cm}
    \textcolor{sidenotegray}{\LARGE General guidelines}\\
    \vspace{5.0cm}
    \textcolor{sidenotegray}{\large \today}
    \vfill
}
\author{}
\date{}

\begin{document}

\maketitle
\thispagestyle{empty}
\newpage
\tableofcontents
\newpage
\section{Overview}

\subsection{Module Objectives}
\sn{Learning goals}

This module covers:
\begin{itemize}
    \item Repository organization.
    \item Available scripts.
    \item General pipeline structure.
    \item Modules organization.
\end{itemize}

\section{Background}

\subsection{General Context}
\sn{Why this matters}

Key concepts:
\begin{itemize}
    \item The integration between biological knowledge and IT or
        bioinformatics tools can be harsh.
    \item The presence and interference of multiple key concepts of each area
        creates a big wall on newcomers.
    \item Explain key features and provide support in a organized way can help
        to circumvent this initial barrier.
\end{itemize}

\section{Workflow \& Methods}

\subsection{The repository}
\sn{Key parameters}
The repository is structured aiming to facilitate user operation and base codes
and modules maintenance. The main separation happens in the global folder, that
stores templates, bibliography files and scripts for project managers usage.
\begin{verbatim}
genomes_workflow_lbcm_ufrgs
|-- CHANGELOG.md
|-- README.md
|-- global
|   |-- bibliography
|   |-- scripts
|   `-- templates
|-- modules
|   `-- 00_introduction
|-- requirements.txt
|-- scripts
|   |-- README.md
|   |-- build_course.py
|   |-- config.yaml
|   |-- interactive_builder.py
|   |-- main.py
|   |-- requirements.txt
|   `-- utils
`-- setup.py
\end{verbatim}

\subsection{Available scripts}
\subsubsection{Global Scripts}
\begin{itemize}
    \item \textbf{main.py:} Terminal choice selection to execute the possible
        tools scripts, all related to repository creation and maintenance.
    \item \textbf{tools/create\_module.py:} Creates a new module under the
        \textbf{modules/} folder, based on the available template file named
        main\_template.tex and preamble\_template.tex. Also creates a base
        README.md file and an images folder. Receives the name of the module to
        be created(folder name, supposed to be short and not formated), the
        title as intended to be displayed to the final user and a subtitle, if
        needed. 
    \item \textbf{tools/set\_vim\_config.py:} Changes the current .vimrc file to
        a custom one, that can be found on the \textbf{global/templates/}
        folder. Includes UltiSnips and VimTeX shortcuts and QOL settings for
        module production.
\end{itemize}
\subsubsection{Scripts for General Users}
\begin{itemize}
    \item \textbf{build\_course.py:} General command line interface for course
        building. Accepts single modules, complete courses or custom module
        selection. Imports configuration from the config.yaml file and receives
        user arguments depending on final intention.
    \item \textbf{interactive\_builder.py:} Similar in function to the script
        above, although it provides a simple interface to facilitate user usage
        and interaction.
\end{itemize}
\subsection{General pipeline structure}
WORK IN DEVELOPMENT
\subsection{Modules organization}
\begin{description}
    \item[00 - Introduction] - Similar to the main README.md file, this module
        aims to give a general perception on the repository and what composes
        it. Also provides some general explanation on scripts and modules
       currently implemented.
    \item[01 - General guidelines and workflow organization] - This module
        presents a short introduction to Linux logic and terminal usage, as well
        as conda environment creation and logic. Lastly approaches some good
        practices on workflow organization.
\end{{description}
\section{Best Practices}
\begin{itemize}
    \item \textbf{Data backup:} Always keep original data copies, especially
        when working on sequenced data.
    \item \textbf{Version control:} Track analysis versions and parameters if
        working on the repository construction and maintenance.
    \item \textbf{Documentation:} Record all analysis steps and decisions if
        working directly on the project.
    \item \textbf{Reproducibility:} Use consistent environments, seeds and
        terminology, referencing external bibliography when necessary.
\end{itemize}

% BIBLIOGRAPHY
\newpage
\printbibliography

\end{document}
