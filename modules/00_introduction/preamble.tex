% Beautiful Notes Preamble - ONLY FOR DAILY RESEARCH NOTES
% NotesTeX-inspired aesthetic design with sidenotes and colored boxes
% DO NOT use for academic papers - use paper_preamble.tex instead

% Basic packages
\usepackage[utf8]{inputenc}
\usepackage[T1]{fontenc}
\usepackage{lmodern}
\usepackage{microtype}
\usepackage[english]{babel}

% WIDE MARGINS FOR SIDENOTES (Notes-specific feature)
\usepackage[
    left=2cm,
    right=6cm,
    top=2cm,
    bottom=2cm,
    marginparwidth=4cm,
    marginparsep=0.5cm
]{geometry}

% Mathematics packages
\usepackage{amsmath}
\usepackage{amssymb}
\usepackage{amsthm}
\usepackage{mathtools}
\usepackage{physics}
\usepackage{siunitx}

% Graphics and figures
\usepackage{graphicx}
\usepackage{float}
\usepackage{caption}
\usepackage{subcaption}
\usepackage{tikz}
\usepackage{pgfplots}
\pgfplotsset{compat=1.18}

% BIBLIOGRAPHY SUPPORT FOR NOTES
\usepackage[style=authoryear,backend=biber,natbib=true]{biblatex}

% Add bibliography resources (notes can reference paper's bibliography)
\addbibresource{../references.bib}
\addbibresource{../../../global/bibliography/master.bib}

% NOTES-SPECIFIC COLORS
\usepackage{xcolor}
\definecolor{noteblue}{RGB}{0,102,204}
\definecolor{sidenotegray}{RGB}{128,128,128}
\definecolor{theoremblue}{RGB}{230,240,255}
\definecolor{definitiongreen}{RGB}{240,255,240}
\definecolor{remarkyellow}{RGB}{255,255,230}

% BEAUTIFUL THEOREM BOXES (Notes-specific feature)
\usepackage{tcolorbox}
\tcbuselibrary{theorems}

\newtcolorbox{definitionbox}{
    colback=definitiongreen,
    colframe=green!50!black,
    title=Definition,
    fonttitle=\bfseries
}

\newtcolorbox{theorembox}{
    colback=theoremblue,
    colframe=blue!50!black,
    title=Theorem,
    fonttitle=\bfseries
}

\newtcolorbox{lemmabox}{
    colback=theoremblue,
    colframe=blue!50!black,
    title=Lemma,
    fonttitle=\bfseries
}

\newtcolorbox{proofbox}{
    colback=white,
    colframe=black!20,
    title=Proof,
    fonttitle=\bfseries
}

\newtcolorbox{remarkbox}{
    colback=remarkyellow,
    colframe=orange!50!black,
    title=Remark,
    fonttitle=\bfseries
}

\newtcolorbox{examplebox}{
    colback=white,
    colframe=gray!50,
    title=Example,
    fonttitle=\bfseries
}

% SIDENOTES (Notes-specific feature)
\usepackage{marginnote}
\renewcommand*{\marginfont}{\footnotesize\color{sidenotegray}}

% Custom sidenote command
\newcommand{\sn}[1]{\marginnote{\raggedright #1}}

% Custom margin note command
\newcommand{\mn}[1]{\marginpar{\footnotesize\color{sidenotegray}\raggedright #1}}

% BEAUTIFUL HEADERS (Notes-specific styling)
\usepackage{fancyhdr}
\setlength{\headheight}{14pt}
\pagestyle{fancy}
\fancyhf{}
\fancyhead[L]{\textcolor{noteblue}{\textbf{01 - Introduction}}}
\fancyhead[R]{\textcolor{noteblue}{\today}}
\fancyfoot[C]{\textcolor{noteblue}{\thepage}}
\renewcommand{\headrulewidth}{0.5pt}
\renewcommand{\footrulewidth}{0.5pt}

% COLORFUL SECTION STYLING (Notes-specific)
\usepackage{titlesec}
\titleformat{\section}
  {\Large\bfseries\color{noteblue}}
  {\thesection}{1em}{}
\titleformat{\subsection}
  {\large\bfseries\color{noteblue}}
  {\thesubsection}{1em}{}

% Code listings with beautiful styling
\usepackage{listings}

\lstdefinestyle{notesstyle}{
    backgroundcolor=\color{gray!10},
    commentstyle=\color{green!50!black},
    keywordstyle=\color{blue},
    numberstyle=\tiny\color{gray},
    stringstyle=\color{red!50!black},
    basicstyle=\ttfamily\footnotesize,
    breakatwhitespace=false,
    breaklines=true,
    captionpos=b,
    keepspaces=true,
    numbers=left,
    numbersep=5pt,
    showspaces=false,
    showstringspaces=false,
    showtabs=false,
    tabsize=2
}
\lstset{style=notesstyle}

% Hyperlinks
\usepackage{hyperref}
\hypersetup{
    colorlinks=true,
    linkcolor=noteblue,
    citecolor=red,
    urlcolor=noteblue,
    pdftitle={Genome Workflow - Base Guide},
    pdfauthor={LBCM UFRGS},
    pdfsubject={General guidelines for working with genomes in the LBCM},
    pdfkeywords={genomics, workflow, LBCM, UFRGS}
}

% RESEARCH-SPECIFIC COMMANDS (Notes only)
\newcommand{\lecture}[2]{\section{#1 - #2}}
\newcommand{\reading}[2]{\subsection{Reading: #1}\sn{#2}}
\newcommand{\idea}[1]{\textbf{\textcolor{purple}{[IDEA]: #1}}}
\newcommand{\todo}[1]{\textbf{\textcolor{red}{[TODO]: #1}}}
\newcommand{\important}[1]{\textbf{\textcolor{red}{#1}}}
\newcommand{\highlight}[1]{\textbf{\textcolor{noteblue}{#1}}}

% Math shortcuts
\newcommand{\R}{\mathbb{R}}
\newcommand{\N}{\mathbb{N}}
\newcommand{\Z}{\mathbb{Z}}
\newcommand{\Q}{\mathbb{Q}}
\newcommand{\C}{\mathbb{C}}
\newcommand{\eps}{\varepsilon}
\newcommand{\del}{\partial}

% NOTES-SPECIFIC ENVIRONMENTS (match template exactly)
\newenvironment{progress}{
    \section{Today's Progress}
    \begin{itemize}
}{\end{itemize}}

\newenvironment{ideas}{
    \section{Ideas \& Insights}
    \begin{itemize}
}{\end{itemize}}

\newenvironment{questions}{
    \section{Research Questions}
    \begin{itemize}
}{\end{itemize}}

\newenvironment{references}{
    \section{References}
    \begin{itemize}
}{\end{itemize}}

\newenvironment{notes}{
    \section{Research Notes}
}{}

% THEOREM ENVIRONMENTS using amsthm (match template exactly)
\theoremstyle{definition}
\newtheorem{definition}{Definition}[section]
\newtheorem{theorem}{Theorem}[section]
\newtheorem{lemma}{Lemma}[section]
\newtheorem{example}{Example}[section]
\newtheorem{remark}{Remark}[section]

% Proof environment
\renewenvironment{proof}[1][\proofname]{%
    \par\pushQED{\qed}%
    \normalfont\topsep6\p@\@plus6\p@\relax
    \trivlist\item[\hskip\labelsep\bfseries #1\@addpunct{.}]\ignorespaces
}{%
    \popQED\endtrivlist\@endpefalse
}

% Date formatting
\usepackage{datetime}
\newdateformat{notedate}{\THEDAY\ \monthname[\THEMONTH] \THEYEAR}
